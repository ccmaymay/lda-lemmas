\documentclass[11pt,letterpaper]{article}
\usepackage{emnlp2016}
\usepackage{times}
\usepackage{latexsym}

% Uncomment this line for the final submission:
%\emnlpfinalcopy

%  Enter the EMNLP Paper ID here:
\def\emnlppaperid{835}

% To expand the titlebox for more authors, uncomment
% below and set accordingly.
% \addtolength\titlebox{.5in}

%\usepackage[utf8]{inputenc} % allow utf-8 input
%\usepackage[T1]{fontenc}    % use 8-bit T1 fonts
%\usepackage{hyperref}       % hyperlinks
%\usepackage{url}            % simple URL typesetting
%\usepackage{booktabs}       % professional-quality tables
%\usepackage{amsfonts}       % blackboard math symbols
%\usepackage{nicefrac}       % compact symbols for 1/2, etc.
%\usepackage{microtype}      % microtypography

%\usepackage{natbib}
\usepackage[ruled]{algorithm2e}

\usepackage[fleqn]{amsmath}
\usepackage{amsfonts}
\usepackage{amsbsy}
\usepackage{xspace}
\usepackage{graphicx}

\usepackage{rotating}
\usepackage{times}
%\usepackage{url}                                                       
\makeatletter                                                           
\newcommand{\@BIBLABEL}{\@emptybiblabel}                                
\newcommand{\@emptybiblabel}[1]{}                                       
\makeatother                                                            
\usepackage[hidelinks]{hyperref}
\usepackage{latexsym}
\usepackage{appendix}
\usepackage{booktabs}
%\usepackage[small]{caption}
%\usepackage{subcaption}
\usepackage{multirow}
\usepackage{rotating}

\usepackage[inline,shortlabels]{enumitem}
\setlist*[enumerate,1]{label=$(\arabic*)$}
\setlist*[itemize,1]{label=$\bullet$}

\usepackage[T1]{fontenc}
\usepackage[utf8]{inputenc}
\usepackage{amsmath,amssymb}
\usepackage[russian,english]{babel}

\usepackage{cleveref}
\crefname{section}{§}{§§}
\Crefname{section}{§}{§§}
\crefname{figure}{Fig}{}
\crefname{algorithm}{Alg.}{}
\crefformat{section}{§#2#1#3}  % remove space between section symbol and the number


\long\def\devour#1{\ignorespaces}

\renewcommand{\vec}{\boldsymbol}   % optional
\newcommand{\vf}{{\vec{f}}}
\newcommand{\vg}{{\vec{g}}}
\newcommand{\vh}{{\vec{h}}}
\newcommand{\vx}{{\vec{x}}}
\newcommand{\vz}{{\vec{z}}}
\newcommand{\vm}{{\vec{m}}}
\newcommand{\vw}{{\vec{w}}}
\newcommand{\vc}{{\vec{c}}}
\newcommand{\vy}{{\vec{y}}}
\newcommand{\vl}{{\vec{l}}}
\newcommand{\vs}{{\vec{s}}}
\newcommand{\vphi}{{\vec{\phi}}}
\newcommand{\valpha}{{\vec{\alpha}}}
\newcommand{\vxi}{{\vec{\xi}}}

\newcommand{\vmu}{{\vec{\mu}}}
\newcommand{\vtheta}{{\vec{\theta}}}
\newcommand{\veta}{{\vec{\eta}}}
\newcommand{\vomega}{{\vec{\omega}}}
\newcommand{\vpsi}{{\vec{\psi}}}
\newcommand{\vbeta}{{\vec{\beta}}}

\newcommand{\dirac}[2]{\delta_{#1}\left(#2\right)}
\newcommand{\Discrete}{\ensuremath{\mathrm{Discrete}}}
\newcommand{\Dirichlet}{\ensuremath{\mathrm{Dirichlet}}}

\newcommand{\DR}{\ensuremath{\textrm{DR}}}

% reduce space before \paragraph command -- http://tex.stackexchange.com/questions/219676/how-to-change-spacing-before-and-after-paragraph-command/219677#219677
%\makeatletter
%\renewcommand\paragraph{\@startsection{paragraph}{4}{\z@}%
%                                    {3.25ex \@plus1ex \@minus.2ex}%
%                                    {-1em}%
%                                    {\normalfont\normalsize\bfseries}}
%\makeatother

\usepackage{amssymb}
\DeclareMathOperator*{\Expect}{\mathop{\mathbb{E}}}
\DeclareMathOperator*{\argmax}{\mathop{\text{argmax}}}


% new math commands

\usepackage{color}
\usepackage{xcolor}
\definecolor{darkgrey}{rgb}{0.2,0.2,0.2}
\definecolor{grey}{rgb}{0.9,0.9,0.9}
\definecolor{darkblue}{rgb}{0.0,0.0,0.5}
\definecolor{darkpurple}{rgb}{0.4,0.0,0.4}
\definecolor{darkred}{rgb}{0.5,0.0,0.0}
\definecolor{darkorange}{rgb}{0.5,0.45,0.4}
\definecolor{darkgreen}{rgb}{0.0,0.5,0.0}
\definecolor{darkergreen}{rgb}{0.0,0.4,0.0}
\definecolor{lightblue}{rgb}{0.8,0.8,1.0}
\definecolor{lightgreen}{rgb}{0.8,1.0,0.8}
\definecolor{lightred}{rgb}{1.0,0.8,0.8}
\definecolor{lightyellow}{rgb}{1.0,1.0,0.8}
\definecolor{lightorange}{rgb}{1.0,0.9,0.8}
\definecolor{lightgrey}{rgb}{0.96,0.97,0.98}
\definecolor{brilliantlavender}{rgb}{0.96, 0.73, 1.0}
\definecolor{deeppink}{rgb}{1.0,0.08,0.58}
\definecolor{hotpink}{rgb}{1.0,0.41,0.71}
\definecolor{pink}{rgb}{1.0,0.75,0.80}

% define notes
\usepackage[russian,english]{babel}
\usepackage{soul}
\usepackage{color}
\newcommand{\Note}[3]{\sethlcolor{#2}\hl{[\textbf{#1}: #3]}}
\renewcommand{\Note}[3]{}

\newcommand{\timv}[1]{\Note{timv}{brilliantlavender}{#1}}
\newcommand{\ryan}[1]{\Note{ryan}{lightorange}{#1}}
\newcommand{\chandler}[1]{\Note{chandler}{pink}{#1}}
\newcommand{\todo}[1]{\Note{todo}{red}{#1}}

% space hacks
\setlength\belowcaptionskip{-15pt}
\setlength\titlebox{2cm}
\usepackage[hang,flushmargin]{footmisc} 


\newcommand{\defn}[1]{{\bf #1}}%

%i commented this out because i cannot compile this option
%what's the easiest way to install this on my mac?
%\usepackage[subtle]{savetrees}

\newcommand{\software}[1]{\textsc{#1}}
\newcommand{\word}[1]{{\em #1}}
\newcommand{\lemma}[1]{\textsc{#1}}
\newcommand{\att}[2]{{\small \texttt{#1}}=\textsc{#2}}
\newcommand{\gloss}[1]{``#1''}
\newcommand{\UNK}{\textsc{unk}\xspace}

%\newcommand{\func}{\mathcal{E}(t, s, u, w)}
%\newcommand{\grad}{\nabla_{\vtheta}\,\func}
%\newcommand{\funcw}{\mathcal{E}(t, s, u)}
%\newcommand{\gradw}{\nabla_{\vtheta}\,\funcw}

\newcommand{\wfa}{\textsuperscript{$\star$}\xspace}
\newcommand{\wfb}{\textsuperscript{$\dagger$}\xspace}
\newcommand{\wfc}{\textsuperscript{$\ddagger$}\xspace}
\newcommand{\wfd}{\textsuperscript{$\star\star$}\xspace}

\def\figref#1{Figure~\ref{fig:#1}}
\def\figlabel#1{\label{fig:#1}\label{p:#1}}
\def\Tabref#1{Table~\ref{tab:#1}}
\def\tabref#1{Table~\ref{tab:#1}}
\def\tablabel#1{\label{tab:#1}\label{p:#1}}
\def\Secref#1{Section~\ref{sec:#1}}
\def\secref#1{Section~\ref{sec:#1}}
\def\seclabel#1{\label{sec:#1}\label{p:#1}}
\def\eqref#1{Eq.~\ref{eqn:#1}}
\def\eqrefn#1{\ref{eqn:#1}}
\def\eqsref#1#2{Eqs.~\ref{eqn:#1}-\ref{eqn:#2}}
\def\eqlabel#1{\label{eqn:#1}}
\def\subsp#1{P_{\mbox{{\scriptsize\rm #1}}}}


% space hacks
%\setlength\belowcaptionskip{-4pt}
%\setlength\titlebox{2cm}

% You can expand the titlebox if you need extra space
% to show all the authors. Please do not make the titlebox
% smaller than 5cm (the original size); we will check this
% in the camera-ready version and ask you to change it back.

%\titlespacing\section{0pt}{1pt plus 4pt minus 4pt}{12pt plus 4pt minus 4pt}
%\titlespacing\subsection{0pt}{1pt plus 0pt minus 0pt}{1pt plus 0pt minus 0pt}
%\titlespacing\subsubsection{0pt}{1pt plus 0pt minus 0pt}{1pt plus 0pt minus 0pt}

\title{Analysis of Morphology in Topic Modeling}

\begin{document}
\maketitle
\begin{abstract}
    Topic models make strong assumptions about their data.  In
    particular, different words are implicitly assumed to
    have different meanings: topic models are often used as
    human-interpretable dimensionality reductions and a proliferation
    of words with identical meanings would undermine the utility of the
    top-$m$ word list representation of a topic.  Though a number
    of authors have added preprocessing steps such as lemmatization to
    better accommodate these assumptions, the effects of such data
    massaging have not been publicly studied.  We make first steps
    toward elucidating the role of morphology in topic modeling by
    testing the effect of lemmatization on the interpretability of a
    latent Dirichlet allocation (LDA) model.  Using a word intrusion
    evaluation, we quantitatively demonstrate that lemmatization
    provides a significant benefit to the interpretability of a model
    learned on Wikipedia articles in a morphologically rich language.
\end{abstract}


\section{Introduction}\label{sec:introduction}

Topic modeling is a standard tool for unsupervised analysis of large
text corpora. At the core, almost all topic models pick up on
co-occurrence signals between different words in the corpus, that is,
words that occur often in the same sentence, are likely to belong to
the same latent topic. In languages that exhibit rich inflectional
morphology, the signal becomes weaker given the proliferation of
unique tokens. In this work, we explore the effect of token-based
lemmatization on the performance of topic models.

Syntactic information is not generally considered to exert a strong
force on the thematic nature of a document.  Indeed, for this reason
topic models often make a bag-of-words assumption, discarding the order
of words within a document.  In morphologically rich languages,
however, syntactic information is often encoded in the word form
itself.  This kind of syntactic information is a nuisance variable
in topic modeling and is prone to polluting a topic representation
learned from data~\cite{boydgraber2014}.
For example, consider the Russian name
{\em Putin}; in English, we have a single type that represents in the
concept in all syntactic contexts, whereas in Russian
{\selectlanguage{russian} Путин} appears with various inflections,
e.g., {\selectlanguage{russian}Путина},
{\selectlanguage{russian}Путину}, {\selectlanguage{russian}Путине},
and {\selectlanguage{russian}Путином}. Which form of the name one uses
is fully dependent on the syntactic structure of the sentence. Compare
the utterances {{\selectlanguage{russian}мы говорим о Путине} ({\em we
    are speaking about Putin}) and {{\selectlanguage{russian}мы
      говорим Путину} ({\em we are speaking to Putin})}: both sentences
  are thematically centered on Putin, but two different word forms
  are employed.
English stop words like prepositions often end up as inflectional
suffixes in Russian, so lemmatization on Russian performs some
of the text normalization that stop word filtering performs on English.
Topic models are generally sensitive to stop
words in English~\cite{wallach2009,blei2010,eisenstein2011}, hence we
expect them to be sensitive to morphological variation in languages
like Russian.

  \ryan{Make connection here to stop words. We filter stop words (or try to) in topic models, so we should
    filter these endings as well?}
\chandler{I don't understand, what is the connection to stop words?}
\ryan{English stop words like prepositions are often inflectional suffixes in Russian. In a sense, lemmatization is analogous to filtering stop words. Consider
the Putin example: ``to'' is a stop word, but that is fully expressed by a suffix in Russian.}
\chandler{oooh, okay.  I modified this paragraph and in particular
    repurposed your comment at the end of the
    paragraph, change as desired.  I think this addresses your initial
    question (for the sake of the paper).}
\ryan{I love it!}

In this study, we show that
\begin{itemize}
    \item truncated documents, imitating the sparsity seen in social
        media, reduce interpretability;
    \item if lemmatization is used, a filtered vocabulary yields more
        interpretable topics than an informative prior; and
    \item overall, interpretability is best when the corpus consists
        of long documents, the vocabulary is filtered, and lemmatization
        is applied.
\end{itemize}

\section{Morphology and Lemmatization}\label{sec:inflectional}

Morphology concerns itself with the internal structure of individual
words.  Specifically, we focus on {\em inflectional morphology}, word
internal structure that marks syntactically relevant linguistic
properties, e.g., person, number, case and gender on the word form
itself. While inflectional morphology is minimal in English and
virtually non-existent in Chinese, it occupies a prevalent position in
many languages' grammars, e.g., Russian. In fact, Russian will often
express relations marked in English with prepositions, simply through
the addition of a suffix, often reducing the number of words in a
given sentence. The collection of inflections of the same stem is preferred to as a
paradigm.  The Russian noun, for example, forms a paradigm with 12
forms.  See the sample paradigm in Table~\ref{tab:paradigm} for an
example.\footnote{Note that Table~\ref{tab:paradigm} contains several
  entries that are identical, e.g., the singular genitive is the same
  as the singular accusative. This is a common phenomenon known as
  syncretism \cite{baerman2005syntax}, but it is not universal over all nouns---plenty of other
  Russian nouns {\em do} make the distinction between
  genitive and accusative in the singular.} The Russian verb is even more expressive with more
than 30 unique forms \cite{wade2010comprehensive}.

In the context of NLP, large paradigms imply an increased token to type
ratio, greatly increasing the number of unknown words. One method to
combat this issue is to {\em lemmatize} the sentence.  A lemmatizer maps each
inflection (an element of the paradigm) to a canonical form known as
the lemma, which is typically the form found in dictionaries written
in the target language.
\todo{example of lemma mapping here}
In this work, we employ the TreeTagger
lemmatizer \cite{schmid1994probabilistic}.\footnote{
   {\tiny  \url{http://www.cis.uni-muenchen.de/~schmid/tools/TreeTagger/}.
}}
The parameters were estimated using the Russian corpus described in
\newcite{sharov2011proper}.

\begin{table}
  \begin{tabular}{l | l l }
    & {\bf Singular} & {\bf Plural} \\ \hline
    {\bf Nominative} &  {\selectlanguage{russian}пес} ({\em pyos}) & {\selectlanguage{russian}псы}    ({\em psy})   \\
    {\bf Genitive} &  {\selectlanguage{russian}пса} ({\em psa}) & {\selectlanguage{russian}псов}    ({\em psov})  \\
    {\bf Accusative} &  {\selectlanguage{russian}пса} ({\em psa}) & {\selectlanguage{russian}псов}    ({\em psov})  \\
    {\bf Dative} &  {\selectlanguage{russian}псу} ({\em psu}) & {\selectlanguage{russian}псам}    ({\em psam})  \\
    {\bf Locative} &  {\selectlanguage{russian}псе} ({\em psye}) & {\selectlanguage{russian}псах}   ({\em psax})  \\
    {\bf Instrumental} &  {\selectlanguage{russian}псом} ({\em psom}) & {\selectlanguage{russian}псами}  ({\em psami}) \\
  \end{tabular}
  \caption{A inflectional paradigm for the Russian word
    {\selectlanguage{russian}пес} ({\em pyos}), meaning ``dog''.  Each
    of the 12 different entries in the table occurs in a distinct
    syntactic context. A lemmatizer canonicalizes these forms to
    single form, which is the nominative singular in, reducing the sparsity present in the corpus.}
    \label{tab:paradigm}
\end{table}



\section{Related Work}\label{sec:related-work}
%\ryan{I can probably do the non-topic-modeling part of this section.}
%\chandler{Please do, and make this a priority, after the ingest/lemmatizer details---the related work in topic modeling is non-existent.}

Though applied in many
studies~\cite{deerwester1990,hofmann1999,mei2007,nallapati2008,lin2009},
lemmatization has not been directly explored in the context of topic
modeling.  An infinite-vocabulary LDA model containing a prior on words
similar to an $n$-gram model has been developed~\cite{zhai2013}; this
prior could be viewed as loosely encoding beliefs of a
concatenative morphology, but its effect was not analyzed in
isolation.

To measure the effect of lemmatization on topic models we must first
define ``topic model.''  In this study, for comparability with other
work, we restrict our attention to latent Dirichlet allocation
(LDA)~\cite{blei2003}, the canonical Bayesian graphical topic model.
We want to measure the performance of a topic model by its
interpretability, as topic models are best suited to discovering
human-interpretable decompositions of the data~\cite{may2015}.
We note there are more modern but less widely-used topic models than
LDA such as the sparse additive generative
(SAGE) topic model, which explicitly models the background word
distribution and encourages sparse topics~\cite{eisenstein2011}, or the
nested hierarchical Dirichlet process (nHDP) topic model, which
represents topics in a hierarchy and automatically infers its effective
size~\cite{paisley2015}.  These models may render more interpretable
results overall.  However, we are currently interested in the
\emph{relative} impact of lemmatization on a topic model, we are
unaware of any direct prior work, and we wish for our results to be
widely applicable across research and industry.  Thus we leave these
alternative topic models as considerations for future work.

While not satisfactorily explored in the topic modeling community,
morphology has been actively investigated in the context of
word-embeddings.  The latent topic vectors that topic models discover
have many parallels to continuous embeddings---both are real-valued
representations that stand proxy for (some notion of) lexical semantic
information. Most notably, \newcite{BianGL14} learned
embeddings for individual morphemes jointly within the standard {\sc word2vec}
model \cite{mikolov2013distributed} and \newcite{SoricutO15} used the embeddings
themselves to induce morphological analyzers. Character-level embedding approaches
have also been explored with the express aim of capturing morphology \cite{santos2014learning,LingDBTFAML15}.

\ryan{Add self citation to NAACL 2015 in camera ready. Keep out for anonymity.}
\todo{mention word sense disambiguation applications of topic models?
    they rely more heavily on lemmatization but use topic models (to
    some degree) as substitutes for models of word sense...}
\ryan{So do WSD people use lemmatization before they apply LDA to the task?}

\begin{table*}
    \centering
    \begin{tabular}{l|l}
        view & topic \\\hline

        lem & {\selectlanguage{russian}деревня\wfa сельский поселение пункт сельсовет} \\
        non & {\selectlanguage{russian}деревня\wfa деревни\wfa деревне\wfa жителей волости} \\\hline

        lem & {\selectlanguage{russian}клетка лечение\wfa заболевание\wfb препарат действие} \\
        non & {\selectlanguage{russian}лечения\wfa течение лечение\wfa крови заболевания\wfb} \\\hline

        lem & {\selectlanguage{russian}японский\wfa япония\wfb корея префектура смотреть} \\
        non & {\selectlanguage{russian}считается японии\wfb японский\wfa посёлок японской\wfa} \\\hline

        lem & {\selectlanguage{russian}художник\wfa искусство\wfb художественный\wfa картина\wfc выставка\wfd} \\
        non & {\selectlanguage{russian}искусства\wfb музея картины\wfc выставки\wfd выставка\wfd} \\\hline
    \end{tabular}
    \caption{Manually-aligned topic pairs: the first topic in each pair
        is from the lemmatized model, the second pair is a semantically
        similar topic in the non-lemmatized model.  Within each pair,
        each of the symbols \wfa, \wfb, \wfc, and \wfd (separately)
        denotes word forms of a shared lemma.
        The lemmatized topic representations are more
        diverse than those of the non-lemmatized topic representations.
        For example, the non-lemmatized version of the first topic
        contains three inflections of the Russian word
        {\selectlanguage{russian}деревня} ({\em village})---successive
        inflectional forms add little or no information to the topic.
        \todo{more explanations}
    }
    \label{tab:topics}
\end{table*}

\section{Experiments}\label{sec:experiments}

For some pre-specified
number of topics $K$ and Dirichlet concentration hyperparameters
$\veta$ and $\valpha$, the LDA topic model represents a vocabulary as a
set of $K$ i.i.d.\ topics $\vbeta_k$, represents each document as a
an i.i.d.\ mixture over those topics (with mixture weights
$\vtheta_d$), and specifies that each token in a document is
generated by sampling a word type from the document's topic mixture:
\begin{align*}
    \vbeta_k  & \sim \Dirichlet\left(\veta\right) \\
    \vtheta_d & \sim \Dirichlet\left(\valpha\right) \\
    z_{d,n}              & \sim \Discrete\left(\vtheta_d\right) \\
    w_{d,n}              & \sim \Discrete\left(\vbeta_{z_{d,n}}\right)
\end{align*}

Meaningful evaluation of topic models is notoriously
difficult and has received considerable attention in the
literature~\cite{chang2009,wallach2009a,newman2010,mimno2011}.
In general we desire an evaluation metric that correlates with a
human's ability to use the model to explore or filter a large dataset,
hence, the interpretability of the model.  In this study we moreover
require an evaluation metric that is comparable across different views
of the same corpus.

With those concerns in mind we choose a \emph{word intrusion}
evaluation:
a human expert is shown one topic at a time, represented
by its top $m$ words (for some small number $m$) in random order, as
well as an additional word (called the \emph{intruder}) randomly placed
among the $m$ topic words~\cite{chang2009}.
The intruder is randomly selected from the set of high-probability
words from other topics in the model.
The expert is tasked with identifying the intruder in each list of
$m + 1$ words.
As in prior work~\cite{chang2009}, we instruct the expert to ignore
syntactic and morphological patterns.
%Examples from our word
%intrusion interface on both the lemmatized and non-lemmatized data
%are depicted in Figure~\ref{fig:word-intrusion}.


\todo{explain all those Russian words in figure/table, or remove}

%\begin{table*}
%    \centering
%    \begin{tabular}{l|l}
%        experiment & topic \\\hline
%
%        trunc + filt & {\selectlanguage{russian}организация основать университет японский деятельность} \\
%        trunc + filt & {\selectlanguage{russian}деревня сельский поселение примерно московский} \\
%        trunc + filt & {\selectlanguage{russian}посёлок тип положение географический япония} \\
%        trunc + filt & {\selectlanguage{russian}современный художник жить греческий искусство} \\
%        trunc + filt & {\selectlanguage{russian}край районный сельсовет поселение сельский} \\\hline
%
%        full & {\selectlanguage{russian}остров японский япония корея префектура} \\
%        full & {\selectlanguage{russian}группа альбом песня the год} \\
%        full & {\selectlanguage{russian}театр музыкальный музыка и роль} \\
%        full & {\selectlanguage{russian}деревня район сельский состав поселение} \\
%        full & of and a in i \\
%        full & {\selectlanguage{russian}и в при с у} \\
%        full & {\selectlanguage{russian}он и она фильм её} \\\hline
%    \end{tabular}
%    \caption{\todo}
%    \label{tab:topics-insignif}
%\end{table*}

If the model is interpretable, the $m$ words from a topic will be
internally coherent whereas the intruder word is likely to stand out.
Thus a model's interpretability can be quantified by the fraction
of topics for which the expert correctly identifies the intruder.  We
call this value the \emph{detection rate}:
\begin{equation*}
    \DR = \frac{1}{K} \sum_{k=1}^K \dirac{i_k}{\omega_k}
\end{equation*}
where $K$ is the number of topics in the model, $i_k$ is the index
of the intruder in the randomized word list generated from topic $k$,
and $\omega_k$ is the index of the word the expert identified as the
intruder.  We note this is just the mean (over topics) of the
\emph{model precision} metric from prior work~\cite{chang2009}
when one expert is used instead of several non-experts.

Our corpus consists of Russian Wikipedia articles from the dump
released on 11/02/2015.\footnote{The Wikipedia dump is from November 11, 2015.}

}
We stripped the XML portion of the formatting and then ran the
lemmatizer described in Section~\ref{sec:inflectional}.  When the
lemmatizer does not recognize a word, we back off to the word form
itself.\footnote{
    11\% of the 378 million tokens in the raw corpus were
    unrecognized by the lemmatizer.
}

We consider two preprocessing schemes to account for stop words and
other high-frequency terms in the corpus.  First, we compute the
vocabulary as the top 10,000 words by document frequency,\footnote{
    Due to minor implementation concerns the lemmatized and
    non-lemmatized vocabularies consist of the top 9387 and 9531 words
    (respectively) by document frequency.
}
separately for the lemmatized and non-lemmatized data, and
specify an asymmetric prior on each document's topic proportions
$\vtheta$.
We refer to this preprocessing scheme
as the \emph{unfiltered-asymmetric} setting.  The second modeling scheme we
consider uses a vocabulary with high-frequency words filtered out and a
uniform prior on the document-wise topic proportions.
(We refer to this setting as \emph{filtered-symmetric}.)
Specifically, a 10,000 word vocabulary is formed from the
lemmatized data by removing the top 100 words by document frequency
over the corpus and taking the next 10,000.  To determine the
non-lemmatized vocabulary, we map the filtered lemmatized
vocabulary onto all word forms that produce one of those lemmas in
the data.  Finally, observing that some of the uninformative
high-frequency words reappear in this projection, we remove any
of the top 100 words from the lemmatized and non-lemmatized corpora
from this list, producing a non-lemmatized vocabulary of 72,641 words.
While the large size of this vocabulary slows learning,
we do not believe it impacts the results negatively;
our priority is retaining the information captured by the lemmatized
vocabulary to provide a fair comparison.

In addition to exploring different choices of vocabulary, we also
consider truncating the documents to their first 50 tokens.\footnote{
    As the vocabulary does not contain rare words, the number of
    tokens per document seen by the model is less than 50.
}
This augmentation simulates data sparsity by reducing the amount of
content-bearing signal in each document, so we might expect the
truncated documents to more greatly benefit from lemmatization (which
can be cast as a dimensionality reduction method).

We learn LDA by stochastic variational
inference~\cite{hoffman2013}, initializing the models randomly and
using fixed priors.\footnote{
    In preliminary experiments Gibbs
    sampling with hyper-parameter optimization did not improve
    interpretability.
}
We specify $K = 100$ topics to all models.
Uniform priors with $\eta_v = 0.1$ and
$\alpha_k = 5 / K$ were given to
filtered-symmetric models; non-uniform priors with
$\eta_v = 0.1$, $\alpha_1 = 5$, and $\alpha_k = 5 / (K-1)$
for $k > 1$
were given to unfiltered-asymmetric models.
The local hyperparameters $\valpha$ are informed by mean
document word usage and document length; in particular, we
believe approximately 50\% of the word tokens in the corpus are
uninformative.

The detection rate for all four configurations (filtered-symmetric or unfiltered-asymmetric
vocabulary and full-length or truncated documents), and the
p-values for one-sided detection rate differences (testing our
hypothesis that the lemmatized models yield higher detection rates than
the non-lemmatized models), are reported in
Table~\ref{tab:detection-rate}.  Word intrusion performance benefits
significantly from lemmatization on a filtered vocabulary and a
symmetric prior.
Truncated documents exhibit lower
performance overall and are helped less by lemmatization (posing
challenges for social media applications).
%This
%result confirms the intuition that successful topic modeling on
%short documents such as those found in social media requires close
%attention to text normalization and filtering.
Further, we observe differences
between use of an asymmetric prior on an unfiltered vocabulary and
use of a symmetric prior on a vocabulary with stop words filtered out.

\begin{table}
  \centering
    \begin{tabular}{rrr|rr|r}
              &              &                & \multicolumn{2}{|c|}{$\DR$} &       p-val   \\\hline
        vocab & prior        & docs           & non         & lem         &         $\Delta$ \\\hline
        unfilt & sym         & full           &          0.54 &          0.52 &          0.61 \\
        \textbf{filt} & \textbf{asym} & \textbf{full} & \textbf{0.50} & \textbf{0.65} & \textbf{0.02} \\
        unfilt & sym         & trunc          &          0.37 &          0.37 &          0.50 \\
        filt & asym          & trunc          &          0.43 &          0.47 &          0.28 \\
    \end{tabular}
    \caption{Detection rate for the non-lemmatized (non) and
        lemmatized (lem) models
        and p-values for the one-sided detection rate difference tests.
        (filt and unfilt indicate whether or not the vocabulary is
        filtered; sym and asym indicate whether the prior is symmetric,
        trunc and full indicate whether the documents are truncated.)
        The detection rate benefits significantly from lemmatization on
        a filtered vocabulary (highlighted in bold).}
    \label{tab:detection-rate}
\end{table}

We find that topics from the unfiltered-asymmetric models often contain
stop words despite the first topic receiving half of the prior
probability mass.  Indeed, many topics consist primarily of stop
words, such as the topic {\selectlanguage{russian}и в при с у}.
Hand-aligned topics from the filtered-symmetric models learned on
full-length documents are shown in Table~\ref{tab:topics}.
There is significant redundancy (multiple inflected word forms of the
same lemma) in the top five words of the non-lemmatized topics; on the
other hand, the diversity of words in the lemmatized topics lends
to human interpretation.

\todo{look at how a given document is represented in various models}




\section{Conclusion}\label{sec:conclusion}

We have demonstrated the impact of lemmatization as a preprocessing
step to LDA on Wikipedia articles in Russian.  In particular, we have
verified the intuition that lemmatization can significantly improve the
interpretability of a topic model.


\bibliographystyle{emnlp2016}
\bibliography{russian}

\end{document}
